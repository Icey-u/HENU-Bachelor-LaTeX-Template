\chapter{绪论}
文档源地址 \href{https://github.com/Icey-u/HENU-Bachelor-LaTeX-Template}{HENU-Bachelor-LaTeX-Template}, 请自行查看是否更新. 

对于理工科来说, \LaTeX 很好的解决了公式排版的问题, 能够让大家把精力放在论文内容而非格式上. 

点名批评河南某大学数学院明明学过 \LaTeX, 却在要写毕业论文时只提供了 Word 模板选项, 因为我不会使用 Word, 所以自己用 \LaTeX 造了个轮子.

本文档为河南大学本科毕业论文 \LaTeX 模板, 适用于 2023 年数学院《毕业论文格式要求》. 下面介绍一下使用方法和其他内容.
\section{使用方法}
主文档使用 \hologo{XeLaTeX} 编译, 若要使用参考文献, 则使用 \hologo{XeLaTeX} => \hologo{BibTeX} => \hologo{XeLaTeX} 编译. 每次目录变动均需编译两次 (\hologo{XeLaTeX}*2)才可.

如果你还不太熟悉 \LaTeX, 那么我建议看 \href{https://liam.page/2014/09/08/latex-introduction/}{一份其实很短的 LaTeX 入门文档} 以及 \href{https://www.ctan.org/pkg/lshort-zh-cn}{lshort-zh-cn}.

下面开始介绍本文档使用方法, 本模板存在两种文档类型;
\begin{lstlisting}
    \documentclass{HENU-Bachelor-laTeX}           % 彩色版
    \documentclass[forprint]{HENU=Bachelor-LaTeX} % 打印版
\end{lstlisting}
其中彩色版有超链接突出显示, 而打印版隐藏了超链接颜色, 建议在交付论文时使用打印版, 避免打印字迹偏淡.